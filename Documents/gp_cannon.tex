\documentclass[12pt, preprint]{aastex}
\bibliographystyle{apj}

% naming macros
\newcommand{\tc}{\textsl{The~Cannon}}
\newcommand{\apogee}{\textsl{APOGEE}}

% math and symbol macros
\newcommand{\set}[1]{\bm{#1}}
\newcommand{\starlabel}{\ell}
\newcommand{\starlabelvec}{\set{\starlabel}}
\newcommand{\mean}[1]{\overline{#1}}
\newcommand{\given}{\,|\,}
\newcommand{\teff}{\mbox{$\rm T_{eff}$}}
\newcommand{\kms}{\mbox{$\rm kms^{-1}$}}
\newcommand{\feh}{\mbox{$\rm [Fe/H]$}}
\newcommand{\xfe}{\mbox{$\rm [X/Fe]$}}
\newcommand{\alphafe}{\mbox{$\rm [\alpha/Fe]$}}
\newcommand{\mh}{\mbox{$\rm [M/H]$}}
\newcommand{\logg}{\mbox{$\rm \log g$}}
\newcommand{\noise}{\sigma_{n\lambda}}
\newcommand{\scatter}{s_{\lambda}}
\newcommand{\pix}{\mathrm{pix}}
\newcommand{\rfn}{\mathrm{ref}}

\begin{document}

\title{\tc\ in the Gaussian Process Framework: \\ Data-driven spectral
model determination}
\author{A.Y.Q.~Ho\altaffilmark{1},
D.~Foreman-Mackey\altaffilmark{2},
David~W.~Hogg\altaffilmark{1,2,3}, 
M.~Ness\altaffilmark{1},
H.-W.~Rix\altaffilmark{1}
}
\altaffiltext{1}{Max-Planck-Institut f\"ur Astronomie, K\"onigstuhl 17, D-69117 Heidelberg, Germany}
\altaffiltext{2}{Center for Cosmology and Particle Physics, Department of Phyics,
New York University, 4 Washington Pl., room 424, New York, NY, 10003, USA}
\altaffiltext{3}{Center for Data Science, New York University, 726 Broadway, 7th Floor, New York, NY 10003, USA}

\email{annaho@mpia.de}

\begin{abstract}

\tc is a data-driven method for determining stellar labels (parameters and 
abundances) from stellar
spectra in the context of vast spectroscopic surveys. Arguably the most 
exciting long-term prospect of \tc is its potential to bring qualitatively
different stellar surveys onto a conssitent stellar parameter and chemical
abundance scale, given a set of reference objects observed in common between
the surveys. In the current framework, a model is fit for that describes
how the flux in each pixel of a continuum-normalized spectrum depends
on the labels of the star. However, a major limitation is the need to select
a spectral model that governs all pixels in the spectrum. In the paper,
the functional form of this model was taken to be quadratic in the labels, 
determined simply through empirical experimentation. A unique set of 
coefficients is determined for each pixel, but the functional form of the 
model is held fixed across pixels. This is, however, not physically motivated:
that model is known to be too flexible for some pixels and too inflexible 
for others. In the Gaussian Process framework, instead of specifying the
functional form of the spectral model, we specify the functional form of the
Gaussian Process model, which in turns optimizes directly for the functional
form of the spectral model.

\end{abstract}

\keywords{}

\section{Introduction}

\tc is a data-driven method for determining stellar labels (parameters and 
abundances) from stellar
spectra in the context of vast spectroscopic surveys. Arguably the most
exciting long-term prospect of \tc is its potential to bring qualitatively
different stellar surveys onto a conssitent stellar parameter and chemical
abundance scale, given a set of reference objects observed in common between
the surveys. In the current framework, a model is fit for that describes
how the flux in each pixel of a continuum-normalized spectrum depends
on the labels of the star.

A major limitation of this existing framework is that the functional form of 
the model is restricted to being the same for each pixel. We took the 
spectral model to be characterized by a coefficient vector $\theta_\lambda$ 
that allowed us to predict the flux at every pixel $f_{n \lambda}$ for a 
given label vector $\textbf{l}_n$:

\begin{equation}
  f_{n\lambda} = g(\starlabelvec_n | \set{\theta}_\lambda) + \mbox{noise} 
\end{equation}

\begin{equation}
  f_{n\lambda} = \set{\theta}_\lambda^T \cdot \starlabelvec_n + \mbox{noise}  
\end{equation}

In this case, we optimize for coefficients of a model that is quadratic in the
labels, such that the label vector is,

\begin{equation}
  \starlabelvec_n \equiv 
  [1, \teff, \logg, \feh, \teff^2, \teff\cdot\logg, \teff\cdot\feh, \logg^2, \logg\cdot\feh, \feh^2]
\end{equation}

A major limitation of this existing framework is that the model is inflexible:
although a set of coefficients is uniquely determined for each pixel, 
the functional form of the model itself is held fixed across pixels. 
This is, however, not physically motivated.  Different pixels should have 
different functional dependencies on labels: for example, one might expect a 
continuum pixel to be linear, while a pixel in a magnesium line should show 
polynomial (?) dependence on temperature. In particular, absorption features,
particularly strong lines, are known to vary non-linearly as a function of 
stellar labels. 

%\section{Data}

%\subsection{Something} 

%\section{Results}

%\section{Discussion}

%\section{Acknowledgements}

%AYQH was partially supported by a Fulbright grant through the German-American
%Fulbright Commission.

%The research has received funding from the European Research Council under the
%European Union's Seventh Framework Programme (FP 7) ERC Grant Agreement n.
%[321035].

%\begin{thebibliography}{24}

%\end{thebibliography}

\end{document}


